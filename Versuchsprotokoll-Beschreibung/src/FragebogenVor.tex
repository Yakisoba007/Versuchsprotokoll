%%This is a very basic article template.
%%There is just one section and two subsections.
\documentclass{article}
%\usepackage[ansinew]{inputenc}
\usepackage[utf8]{inputenc}
\usepackage{german}
\usepackage{geometry}
\usepackage{wasysym} % für die benutzten Symbole
 \geometry{a4paper,left=40mm,right=30mm, top=2cm, bottom=2cm}
 \setlength\parindent{0pt}
\begin{document}


\section*{EMG Messungen}
% nach einem einseitigen Schlaganfall zur Nutzung des Oberflächen-Elektromyographie (sEMG) Signals zur Steuerung von Assistenz- und Rehabilitationsrobotern}

\subsubsection*{Teilnehmer ID:}
\subsubsection*{Versuchsnummer:}
\subsubsection*{Datum und Ort: Feldafing, den 20. August 2013}
\vspace{1cm}

\section*{Fragebogen vor dem Versuch}

\section{Pers\"onliche Angaben}
\begin{center}
  \begin{tabular}{ | p{0.45\textwidth} | p{0.55\textwidth} |  }
    \hline
    Geburtsdatum  & \\ \hline
    Geschlecht &  \Square männlich  \hspace{0.3cm} \Square weiblich\\ \hline
    Gewicht &\\ \hline
    K\"orpergr\"o\ss e &\\ \hline
    H\"andigkeit (bevorzugte Seite) &  \Square rechts  \hspace{0.8cm} \Square links  \hspace{0.8cm} \Square beidseitig \\ \hline
    Beruf &\\ \hline
    Hobbies &\\ \hline
    Sportarten (vor dem Schlaganfall) &\\ \hline
  \end{tabular}
\end{center}

\section{Schlaganfall}
\begin{center}
  \begin{tabular}{ |p{0.45\textwidth} | p{0.55\textwidth} |  }
    \hline
    Wann war Ihr Schlaganfall (Datum)? & \\ \hline
    Auf welcher Hirnseite war der Schlaganfall? & \Square rechts  \hspace{0.8cm} \Square links \\ \hline
    Art des Schlaganfalls & \Square Isch\"amisch  \hspace{0.05cm} \Square H\"amorrhagisch \\ \hline
       In welcher Phase der neurologischen Reha waren Sie nach dem Schlaganfall? & \Square B  \hspace{0.2cm} \Square C  \hspace{0.2cm} \Square D  \hspace{0.2cm} \Square E  \hspace{0.2cm} \Square F  \hspace{2cm} \Square Sonstige: \underline{\hspace{3cm}} \\ \hline 
    Was waren die Auswirkungen des Schlaganfalls? & \\ \hline
    In welcher Phase der neurologischen Rehabilitation sind Sie jetzt? & \Square B  \hspace{0.2cm} \Square C  \hspace{0.2cm} \Square D  \hspace{0.2cm} \Square E  \hspace{0.2cm} \Square F  \hspace{2cm} \Square Sonstige: \underline{\hspace{3cm}} \\ \hline
\end{tabular}
\end{center}

\section{Rehabilitation im Allgemeinen}
\begin{center}
  \begin{tabular}{ |p{0.45\textwidth} | p{0.55\textwidth} |  }
    \hline
    Wieviele Therapiestunden haben Sie pro Woche? & \\ \hline
    Wieviele Stunden davon sind für Ergo- und Physiotherapie vorgesehen? & \\ \hline
    Welche Rehabilitations\"ubungen gefallen Ihnen pers\"onlich am Besten? &\\ \hline
Welchen Rehabilitations\"ubungen empfinden Sie pers\"onlich als am Hilfreichsten? & \\ \hline
Was motiviert Sie in den Therapiesitzungen besonders? & \\ \hline
Was motiviert Sie mehr, \"Ubungen in einem Spiel oder \"Ubungen bei denen Sie Aufgaben des t\"aglichen Lebens meistern? & \Square Spiele \hspace{0.9cm} \Square Aufgaben des t\"aglichen Lebens \\ \hline
Welche F\"ahigkeiten m\"ochten Sie am sehnlichsten wieder erlangen? & \\ \hline
    \end{tabular}
\end{center}


\section{Robotergest\"utzte Rehabilitation}
Wenn Sie bisher keine robotergest\"utzte Therapie erhalten haben, dann reicht es, wenn Sie nur die erste Frage in diesem Abschnitt beantworten.
\begin{center}
  \begin{tabular}{ |p{0.45\textwidth} | p{0.55\textwidth} |  }
  \hline
    Haben Sie bisher roboterunterstützte Reha-Therapie erhalten?  & \Square ja \hspace{1.5cm}\Square nein \\ \hline
    Falls ja, welche Systeme waren das? &  \Square Locomat \hspace{0.3cm}  \Square ARMin \hspace{0.2cm} \Square Sonstige: \underline{\hspace{2cm}}\\ \hline
    Wieviele Trainingseinheiten hatten Sie mit den robotergest\"utzten Systemen?  &\\ \hline
    Wieviel der Therapiezeit wird f\"ur die Vorbereitung und Einstellung des Robotersystems ben\"otigt? & \\ \hline
    Wie empfinden Sie das \"Uben mit einem Robotersystem? %(Sie d\"urfen mehrere ankreuzen)
     & % \Square angenehm \hspace{0.2cm} \Square sehr hilfreich  \hspace{0.2cm} \Square \"uberfl\"ussig \hspace{0.2cm} \Square unheimlich \hspace{0.2cm}  \Square es macht keinen Unterschied \hspace{0.2cm} \Square kompliziert \hspace{0.4cm} \Square Sonstiges: \underline{\hspace{3.3cm}}
 \\ \hline
    Welche Vorteile sehen Sie pers\"onlich in der robotergest\"utzten Therapie? & \\ \hline
    Welche Nachteile sehen Sie pers\"onlich in der roboerst\"utzten Therapie? &  \\ \hline
    Was w\"uerden Sie sich von einem robotergest\"utzten System noch w\"unschen? & \\ \hline
    
  \end{tabular}
\end{center}

\newpage

\section*{Durch den Versuchsleiter auszuf\"ullen}
\section{Einschätzung des EMG-Signals}
\subsection{Links}
\begin{center}
  \begin{tabular}{ |p{0.3\textwidth} |  p{0.1\textwidth} | p{0.1\textwidth} | p{0.1\textwidth} | p{0.1\textwidth} | p{0.1\textwidth} |  }
    \hline
    Muskel & sehr stark & stark & mittel & schwach & sehr schwach \\ \hline
    Brachioradialis  & & & & &\\ \hline
    Flexor Carpi Radialis  & & & & &\\ \hline
    Bizeps unten innen & & & & & \\ \hline
    Bizeps oben au\ss en & & & & &\\ \hline
    Deltoid Front & & & & &\\ \hline
    Deltoid Seite& & & & &\\ \hline
    Pectoralis Major & & & & &\\ \hline
    Teres Minor/Major & & & & &\\ \hline 
  \end{tabular}
\end{center}

\subsection{Rechts}
\begin{center}
  \begin{tabular}{ |p{0.3\textwidth} |  p{0.1\textwidth} | p{0.1\textwidth} | p{0.1\textwidth} | p{0.1\textwidth} | p{0.1\textwidth} |  }
    \hline
    Muskel & sehr stark & stark & mittel & schwach & sehr schwach \\ \hline
    Brachioradialis  & & & & &\\ \hline
    Flexor Carpi Radialis  & & & & &\\ \hline
    Bizeps unten innen & & & & & \\ \hline
    Bizeps oben au\ss en & & & & &\\ \hline
    Deltoid Front & & & & &\\ \hline
    Deltoid Seite& & & & &\\ \hline
    Pectoralis Major & & & & &\\ \hline
    Teres Minor/Major & & & & &\\ \hline 
  \end{tabular}
\end{center}

\end{document}
