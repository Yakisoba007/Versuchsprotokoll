%%This is a very basic article template.
%%There is just one section and two subsections.
\documentclass{article}
%\usepackage[ansinew]{inputenc}
\usepackage[utf8]{inputenc}
\usepackage{german}
\usepackage{geometry}
\usepackage{wasysym} % für die benutzten Symbole
 \geometry{a4paper,left=40mm,right=30mm, top=2cm, bottom=2cm}
 \setlength\parindent{0pt}
\begin{document}


\section*{Versuch zur Bestimmung eines Gütema\ss es der Muskelaktivität}
% nach einem einseitigen Schlaganfall zur Nutzung des Oberflächen-Elektromyographie (sEMG) Signals zur Steuerung von Assistenz- und Rehabilitationsrobotern}

\subsubsection*{Teilnehmer ID:}
\subsubsection*{Versuchsnummer:}
\subsubsection*{Datum und Ort: Feldafing, den 20. August 2013}
\vspace{1cm}

\section*{Fragebogen vor dem Versuch}

\section{Pers\"onliche Angaben}
\begin{center}
  \begin{tabular}{ | p{0.4\textwidth} | p{0.6\textwidth} |  }
    \hline
    Geburtsdatum  & \\ \hline
    Geschlecht &  \Square männlich \Square weiblich\\ \hline
    Alter &\\ \hline
    Gewicht &\\ \hline
    K\"orpergr\"o\ss e &\\ \hline
    H\"andigkeit (bevorzugte Seite) &  \Square rechts \Square links \Square beidseitig \\ \hline
    Beruf &\\ \hline
    Hobbies &\\ \hline
    Sportarten &\\ \hline
  \end{tabular}
\end{center}

\section{Schlaganfall}
\begin{center}
  \begin{tabular}{ |p{0.4\textwidth} | p{0.6\textwidth} |  }
    \hline
    Wann war Ihr Schlaganfall? & \\ \hline
    Auf welcher Seite war der Schlaganfall? & \\ \hline
    Art des Schlaganfalls & \Square Isch\"amisch \Square H\"amorrhagisch \\ \hline
    Welche Probleme hat der Schlaganfall ausgel\"o\ss t? & \\ \hline
    Wie schwer waren Sie von dem Schlaganfall betroffen? & \\ \hline
    In welcher Phase der neurologischen Rehabilitation befinden Sie sich jetzt? & \\ \hline
\end{tabular}
\end{center}

\section{Bisherige Rehabilitation}
\begin{center}
  \begin{tabular}{ |p{0.4\textwidth} | p{0.6\textwidth} |  }
    \hline
    Wieviele Therapiestunden (Physio- und Ergotherapie) haben Sie pro Woche? & \\ \hline
    Haben Sie bisher roboterunterstützte Reha-Therapie erhalten?  & \\ \hline
    Falls ja, welche Systeme waren das? &  \\ \hline
    Benutzen Sie die robotergestützten Systeme regelm\"a\ss ig?  &\\ \hline
    Welche Vorteile sehen Sie in der robotergest\"utzten Therapie? &\\ \hline
    Welche Nachteile sehen Sie in der robotergest\"utzten Therapie? &\\ \hline
    Was w\"uerden Sie sich von einem robotergest\"utzten System w\"unschen? &\\ \hline
    Was erhoffen Sie sich von der Rehabilitation am mei\ss ten? &\\ \hline
  \end{tabular}
\end{center}

\newpage

\section*{Durch den Versuchsleiter auszuf\"ullen}
\section{Initiales EMG-Signal}
\subsection{Links}
\begin{center}
  \begin{tabular}{ |p{0.3\textwidth} |  p{0.1\textwidth} | p{0.1\textwidth} | p{0.1\textwidth} | p{0.1\textwidth} | p{0.1\textwidth} |  }
    \hline
    Muskel & sehr stark & stark & mittel & schwach & sehr schwach \\ \hline
    Brachioradialis  & & & & &\\ \hline
    Flexor Carpi Radialis  & & & & &\\ \hline
    Extensor Carpi Ulnaris  & & & & &\\ \hline
    Bizeps & & & & &\\ \hline
    Trizeps & & & & &\\ \hline
    Deltoid & & & & &\\ \hline
    Pectoralis Major & & & & &\\ \hline
    Teres Minor/Major & & & & &\\ \hline 
  \end{tabular}
\end{center}

\subsection{Rechts}
\begin{center}
  \begin{tabular}{ |p{0.3\textwidth} |  p{0.1\textwidth} | p{0.1\textwidth} | p{0.1\textwidth} | p{0.1\textwidth} | p{0.1\textwidth} |  }
    \hline
    Muskel & sehr stark & stark & mittel & schwach & sehr schwach \\ \hline
    Brachioradialis  & & & & &\\ \hline
    Flexor Carpi Radialis  & & & & &\\ \hline
    Extensor Carpi Ulnaris  & & & & &\\ \hline
    Bizeps & & & & &\\ \hline
    Trizeps & & & & &\\ \hline
    Deltoid & & & & &\\ \hline
    Pectoralis Major & & & & &\\ \hline
    Teres Minor/Major & & & & &\\ \hline 
  \end{tabular}
\end{center}

\end{document}
